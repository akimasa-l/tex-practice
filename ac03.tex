% lualatex ac03
\documentclass{jlreq}
\begin{document}

「何人ものニュートンがいた (There were several Newtons)」
と言ったのは、科学史家ハイルブロンである。% 本だと,とか.の全角カンマや全角ピリオド使ってるけどそれするとVSCodeの拡張機能テキスト校正くんに怒られちゃうので全角句読点を使うことにしました。
同様にコーヘンは「ニュートンは常に2つの貌を持っていた% これも本だと「二つの」ってついてたけどテキスト校正くんに「2つの」に直されてしまった....
(Newton was always ambivalent)」と語っている。

% あと、空行を挟む(コメントも何もついていない)行をつけると段落の区切りになるんだなあ
% 段落の頭は、自動的に全角1文字分の字下げ(インデント)になります。字下げしたくない段落は頭に \noindent をつけます。
\noindent 近代物理学市場でもっとも傑出し、もっとも影響の大きな人物が
ニュートンである事は誰しもが頷くことであろう。
しかしハイルブロンやコーヘンの言うように、
ニュートンはさまざまな、%これも本だと「様々な」だったけどテキスト校正くんに「さまざまな」に直されてしまった...
ときには相矛盾した顔を持ち、
その影響もまた時代とともに大きく変わっていった。


特殊文字の入力:
\# \$ \% \& \_ \{ \} \^{} \~{} \textbackslash

\end{document}